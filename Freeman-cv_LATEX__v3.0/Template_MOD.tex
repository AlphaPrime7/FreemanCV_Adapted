%%%%%%%%%%%%%%%%%%%%%%%%%%%%%%%%%%%%%%%%%
% Freeman Curriculum Vitae
% XeLaTeX Template
% Version 3.0 (September 3, 2021)
%
% This template originates from:
% https://www.LaTeXTemplates.com
%
% Authors:
% Vel (vel@LaTeXTemplates.com)
% Alessandro Plasmati
%
% License:
% CC BY-NC-SA 4.0 (https://creativecommons.org/licenses/by-nc-sa/4.0/)
%
%!TEX program = xelatex
% NOTE: this template must be compiled with XeLaTeX rather than PDFLaTeX
% due to the custom fonts used. The line above should ensure this happens
% automatically, but if it doesn't, your LaTeX editor should have a simple toggle
% to switch to using XeLaTeX.
% 
%%%%%%%%%%%%%%%%%%%%%%%%%%%%%%%%%%%%%%%%%

%----------------------------------------------------------------------------------------
%	PACKAGES AND OTHER DOCUMENT CONFIGURATIONS
%----------------------------------------------------------------------------------------

\documentclass[
	10pt, % Default font size, can be between 8pt and 12pt
]{FreemanCV_MOD}

\columnratio{0.55, 0.45} % Widths of the two columns, specified here as a ratio summing to 1 to correspond to percentages; adjust as needed for your content 

% Headers and footers can be added with the following commands: \lhead{}, \rhead{}, \lfoot{} and \rfoot{}
% Example right footer:
\rfoot{\textcolor{headings}{\sffamily Last update: \today.}}

%----------------------------------------------------------------------------------------

\begin{document}

\begin{paracol}{2} % Begin two-column mode

%----------------------------------------------------------------------------------------
%	YOUR NAME AND CURRICULUM VITAE TITLE
%----------------------------------------------------------------------------------------

\parbox[][0.11\textheight][c]{\linewidth}{ % Box to hold your name and CV title; change the fixed height as needed to match the colored box to the right
	\centering % Horizontally center text
	
	{\sffamily\Huge Tingwei Adeck, MS, MBA} % Your name
	%{\sffamily\Huge Tingwei Adeck, MS, MBA} % Your name
	
	%\medskip % Vertical whitespace
	
	{\cursivefont\Huge\textcolor{headings}{My CV/Resume}}
	
	\vfill % Push content to the top of the box
}

%----------------------------------------------------------------------------------------
%	SUMMARY
%----------------------------------------------------------------------------------------

%\section{SUMMARY}

%{\raggedright\textbf{"Data forever tells the truth even if we fail to extract these truths"-Anonymous}\par}

%\medskip % Extra vertical whitespace before the next section

%----------------------------------------------------------------------------------------
%	WORK EXPERIENCE
%----------------------------------------------------------------------------------------

\section{Work Experience}{\faAlignJustify}

% Each job is added with a \jobentry command. Below is an empty one to use as a template:

%\jobentry
%	{} % Duration
%	{} % FT/PT (full time or part time)
%	{} % Employer
%	{} % Job title
%	{} % Description

% All 5 parameters must be supplied but any can be empty if you don't need them	
	
%------------------------------------------------

\jobentry
	{ OCT 2022 --  AUG 2023} % Duration
	{FT} % FT/PT (full time or part time)
	{Amazon Inc.} % Employer
	{Amazon Area Manager-Ops} % Job title
	\jobdetails
	\begin{itemize}
	\item Leader of approximately 40 associates by shift-planning, running fragmented
plans as conditions change through shift, while ensuring safety.
	 \end{itemize}% Description
	  % Description
	
%------------------------------------------------

\jobentry
	{Fall 2020 -- Fall 2022} % Duration
	{PT} % FT/PT (full time or part time)
	{University of Missouri Kansas City} % Employer
	{Research/Teaching Assistant} % Job title
	\jobdetails
	\begin{itemize}
	\item Spring 2021: Investigated the role of Phospholipase‑C (PLC) on Doubletime(DBT), Period (PER) and BDBT(Bridal DBT) cellular behavior.
	\item Fall 2021: Liposome Flux Assays in the context of Voltage gated channels.
	\item Spring 2022: Investigated interactions of dRSMG and RPSL genes in conferring higher level antibiotic resistance to bacteria.
	 \end{itemize}% Description

%------------------------------------------------

\jobentry
	{Fall 2017 -- Fall 2018} % Duration
	{PT} % FT/PT (full time or part time)
	{West Texas A \& M University} % Employer
	{Undergraduate Research/Teaching Assistant} % Job title
	\jobdetails
	\begin{itemize}
	\item Research in understanding the role of Tph-1 as a mediator in the circadian regulation of gastrointestinal motility and serotonin biosynthesis in Daytime Restricted Fed (DRF) mice.
	 \end{itemize}% Description

%----------------------------------------------------------------------------------------
%	REFERENCES
%----------------------------------------------------------------------------------------

\section{Projects}{\faClipboard}

%\textit{References available on request} % Uncomment if you'd rather not include references and remove the section below

%------------------------------------------------

% This section is laid out using a table. A \tableentry command adds lines with the following parameters:

%\tableentry{Heading}{Content}{spaceafter}
% All 3 parameters must be supplied but any can be empty if you don't need them
% A "spaceafter" value in the third parameter will add some vertical space -- this is to be used between headings, leave it empty for no extra space

%------------------------------------------------
%\medskip % Extra vertical whitespace before the next section

\projectentry
	{Spring 2022} % Duration
	{PT} % FT/PT (full time or part time)
	{University of Missouri Kansas City} % Employer
	{Liposome Flux Assays } % Project Title
	\projectdetails
	\begin{itemize}
	\item Performed lab experiments on liposome flux assays to understand the electrophysiological properties of the bacterial Sodium Voltage gated channel (NavAb).
	\item Data was analyzed using excel and R culminating in a presentation to faculty and peers.
	 \end{itemize}
	 
\projectentry
	{Fall 2022} % Duration
	{FT} % FT/PT (full time or part time)
	{University of Missouri Kansas City} % Employer
	{Parametric and non-parametric analysis of the relationship between family size and health expenses from insurance data} % Project title
	\projectdetails
	\begin{itemize}
	\item Analyzed health insurance data. Details on \href{https://github.com/AlphaPrime7/Health_Insurance_Project_SAS}{Github}.
	\item Uncovered insights on the link between family size and health expense.
	\end{itemize}
	 % Description

%------------------------------------------------
%\medskip % Extra vertical whitespace before the next section

\projectentry
	{Fall 2023} % Duration
	{PT} % FT/PT (full time or part time)
	{Self-Starter} % Employer
	{Normfluodbf} % Project Title
	\projectdetails
	\begin{itemize}
	\item Created the \href{https://www.cran-e.com/package/normfluodbf}{normfluodbf} R package, currently with > 1000 downloads.
	\item Facilitates analysis of pharmaceutical research data by scientists.
	\end{itemize}
	 % Description

%------------------------------------------------
%\medskip % Extra vertical whitespace before the next section



%------------------------------------------------


%----------------------------------------------------------------------------------------

\switchcolumn % Switch to the second (right) column

%----------------------------------------------------------------------------------------
%	COLORED CONTACT DETAILS BOX
%----------------------------------------------------------------------------------------

\parbox[top][0.11\textheight][c]{\linewidth}{ % Box to hold the colored box; change the fixed height as needed to match the box to the left
	\colorbox{shade}{ % Create colored box and specify background color
		\begin{supertabular}{@{\hspace{3pt}} p{0.02\linewidth} | p{0.775\linewidth}} % Start a table with two columns, the table will ensure everything is aligned
			\raisebox{-1pt}{\faHome} & Hedge Lane Terrace, Shawnee, KS 66226 \\ % Address
			\raisebox{-1pt}{\faPhone} & \href{tel:816-562-2583}{+1 (816) 562-2583} \\ % Phone number
			\raisebox{-1pt}{\small\faInbox} & \href{mailto:awesome.tingwei@outlook.com}{meMail} \\ % Email address
			\raisebox{-1pt}{\small\faGithub} & \href{https://github.com/AlphaPrime7}{meGithub} \\ % Github Profile
			\raisebox{-1pt}{\small\faLink} & \href{https://www.cran-e.com/package/normfluodbf}{normfluodbf} \\ % Webpage using ggcircle web link
			\raisebox{-1pt}{\faLinkedinSquare} & \href{https://www.linkedin.com/in/tingwei-adeck/}{meLinkedIn} \\ % LinkedIn profile
			% See fontawesome.pdf in the Fonts folder for all icons you can use
		\end{supertabular}
	}
	\vfill % Push content to the top of the box
}

%----------------------------------------------------------------------------------------
%	EDUCATION
%----------------------------------------------------------------------------------------

\section{Education}{\faGraduationCap}

% Each qualification entry is added with a \qualificationentry command. Below is an empty one to use as a template:

%\qualificationentry
%	{} % Duration
%	{} % Degree
%	{} % Honors, achievements or distinctions (e.g. first class honors)
%	{} % Department
%	{} % Institution

% All 5 parameters must be supplied but any can be empty if you don't need them

%------------------------------------------------

\begin{supertabular}{r l} % Start a table with two columns, the table will ensure everything is aligned

	%------------------------------------------------
	
	\qualificationentry
		{2020 -- 2022} % Duration
		{Master of Science} % Degree
		{} % Honors, achievements or distinctions (e.g. first class honors)
		{Cell and Molecular Biology/Bio-informatics} % Department
		{University of Missouri Kansas City} % Institution
	
	%------------------------------------------------
	
	\qualificationentry
		{2015-2016} % Duration
		{Master of Business Administration} % Degree
		{Beta Gamma Sigma} % Honors, achievements or distinctions (e.g. first class honors)
		{School of Business} % Department
		{West Texas A \& M University} % Institution
	
	%------------------------------------------------
	
	\qualificationentry
		{2016 -- 2018} % Duration
		{Bachelor of Science} % Degree
		{Magna Cum Laude} % Honors, achievements or distinctions (e.g. first class honors)
		{Department of Science} % Department
		{West Texas A \& M University} % Institution
	
	%------------------------------------------------

\end{supertabular}

%----------------------------------------------------------------------------------------
%	AWARDS
%----------------------------------------------------------------------------------------

\section{Honors and Awards}

% This section is laid out using a table. A \tableentry command adds lines with the following parameters:

%\tableentry{Heading}{Content}{spaceafter}
% All 3 parameters must be supplied but any can be empty if you don't need them
% A "spaceafter" value in the third parameter will add some vertical space -- this is to be used between headings, leave it empty for no extra space

%------------------------------------------------

\begin{supertabular}{r l} % Start a table with two columns, the table will ensure everything is aligned

	%------------------------------------------------
	
	\tableentry{2018}{\textbf{Scholastic Endowment Scholarship}}{}
	\tableentry{}{\textit{West Texas A \& M University}}{spaceafter}
	
	%------------------------------------------------
	
	\tableentry{2018}{\textbf{President's Undergraduate Research Program Award}}{}
	\tableentry{}{\textit{West Texas A \& M University}}{spaceafter}
	
	%------------------------------------------------
	
	\tableentry{2016}{\textbf{Golden Key International Honor Society}}{}
	\tableentry{}{\textbf{Founder Scholarship}}{spaceafter}
	\tableentry{}{\textit{West Texas A \& M University}}{spaceafter}
	
	%------------------------------------------------
	
\end{supertabular}

%----------------------------------------------------------------------------------------
%	COMPUTER SKILLS
%----------------------------------------------------------------------------------------

\section{Skills} 

% This section is laid out using a table. A \tableentry command adds lines with the following parameters:

%\tableentry{Heading}{Content}{spaceafter}
% All 3 parameters must be supplied but any can be empty if you don't need them
% A "spaceafter" value in the third parameter will add some vertical space -- this is to be used between headings, leave it empty for no extra space

%------------------------------------------------

\begin{supertabular}{r l} % Start a table with two columns, the table will ensure everything is aligned
	
	%------------------------------------------------
	
	\tableentry{Computer}{SPSS, R, SAS, Python, SQL, Excel, Word}{spaceafter}
	\tableentry{Visualization}{Tableau, PowerBI, R Shiny}{spaceafter}
	
	%------------------------------------------------
	
	\tableentry{Statistics}{T-tests, Anova Family of tests}{}
	\tableentry{}{Chi-sq family of tests, Wilcoxon Tests}{}
	\tableentry{}{Box-Cox Transformation, Data Normalization}{spaceafter}
	
	%------------------------------------------------
	
	\tableentry{Language}{French(fluent), English, Spanish (Duo-fluent)}{spaceafter}
	
	%------------------------------------------------
	
\end{supertabular}

%----------------------------------------------------------------------------------------
%	COMMUNICATION SKILLS
%----------------------------------------------------------------------------------------

\section{Presentations}

% This section is laid out using a table. A \tableentry command adds lines with the following parameters:

%\tableentry{Heading}{Content}{spaceafter}
% All 3 parameters must be supplied but any can be empty if you don't need them
% A "spaceafter" value in the third parameter will add some vertical space -- this is to be used between headings, leave it empty for no extra space

%------------------------------------------------

\begin{supertabular}{r l} % Start a table with two columns, the table will ensure everything is aligned
	
	%------------------------------------------------
	
	\tableentry{Seminars}{Oral Presentations at 3 Rotations Seminars}{}
	\tableentry{}{School of Biological Sciences (UMKC) 2021 -- 2022}{spaceafter}
	
	%------------------------------------------------
	
	\tableentry{Poster}{Poster at the Student Research Conference}{}
	\tableentry{}{West Texas A \& M University -- 2018}{spaceafter}
	
	%------------------------------------------------
	
\end{supertabular}

%----------------------------------------------------------------------------------------
%	SKILLS DESCRIPTION
%----------------------------------------------------------------------------------------

\section{Programs}

\subsection{normfluodbf R package}
An R \href{https://www.cran-e.com/package/normfluodbf}{package} to convert FLUOstar Omega microplate reader dirty ".dbf" file(s) to normalized data frame(s) in a single step.

\subsection{normfluodbf ShinyApp}
A product of the normfluodbf package distributed as a GUI App.

\end{paracol} % End two-column mode

%----------------------------------------------------------------------------------------

\end{document}
%